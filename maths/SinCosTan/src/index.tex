\documentclass{article}
\usepackage{amsmath}

\begin{document}

\title{\Large{Aufgaben zu Sinus, Cosinus und Tangens}}
\author{Sami Hindi}
\maketitle

\section{Sinus}

In einem rechtwinkligen Dreieck beträgt der Winkel $\alpha = 30^\circ$ und die Hypotenuse ist $c = 10$ Einheiten. Berechne die Länge der gegenüberliegenden Seite (Gegenkathete).

\begin{equation}                                                         
\sin \alpha = \frac{Gegenkathete}{Hypotenuse}                            
\end{equation}

\begin{equation}
\sin 30^\circ = \frac{Gegenkathete}{10}
\end{equation}

\begin{equation}
-0.988 = \frac{Gegenkathete}{10}
\end{equation}

\begin{equation}
Gegenkathete = -0.988 \cdot 10
\end{equation}

\begin{equation}
	\underline{\underline{Gegenkathete = -9.88}}
\end{equation}

\section{Cosinus}

In einem Dreieck beträgt der Winkel $\alpha = 45^\circ$, die angrenzende Seite (Ankathete) ist a=8 Einheiten. Finde die Länge der Hypotenuse.

\begin{equation}
	\cos 45^\circ &= \frac{8}{Hypotenuse}
\end{equation}

\begin{equation}
	0.525 &= \frac{8}{Hypotenuse}
\end{equation}

\begin{equation}
	Hypotenuse \cdot 0.525 &= 8
\end{equation}

\begin{equation}
	Hypotenuse &= \frac{8}{0.525}
\end{equation}

\begin{equation}
	Hypotenuse &= 15.28
\end{equation}


\end{document}

